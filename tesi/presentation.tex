\documentclass[aspectratio=169]{beamer}

\usetheme{metropolis}
\setbeamercovered{transparent}
\setbeamertemplate{navigation symbols}{}

\bibliographystyle{alpha}

\definecolor{unibsblue}{RGB}{60, 88, 155}
\definecolor{lightbeige}{RGB}{255, 251, 241}
\definecolor{textcolor}{RGB}{50, 50, 50}

\usepackage{cmbright}
\usepackage[final]{graphicx}
\usepackage{pdfpages}

\graphicspath{{images}}
\usepackage[main=italian,english]{babel}
\usepackage[utf8]{inputenc}
\usepackage{csquotes}
\usepackage[T1]{fontenc}
\usepackage{hyphenat}

%\hyphenation{ATMega16U2}



\title[Relazione Finale]{Sviluppo di un sistema di supporto alla programmazione per dispositivi integrati della famiglia AVR}
\author[S.Fontana Matr. 727199]{Stefano Fontana}
\date{2021/2022}
\makeatletter 
    \newcommand{\insertrawshotauthor}{\beamer@shortauthor} 
    \newcommand{\insertrawshorttitle}{\beamer@shorttitle} 
    \setlength{\metropolis@frametitle@padding}{3.5ex}% <- default 2.2 ex
\makeatother

\defbeamertemplate*{title page}{customized}[1][]
{
    \begin{center}
        \includegraphics[width=.23\textwidth]{logo_unibs_40.png}

        DIPARTIMENTO DI INGEGNERIA DELL'INFORMAZIONE
        
        Corso di Laurea in Ingegneria Informatica
    \end{center}
    \vfill
    \begin{center}
        \insertshorttitle\\
        {\fontfamily{cmr}\fontsize{14}{27}\color{black}\selectfont\textbf{\inserttitle}}
    \end{center}
    \vfill
    \begin{minipage}{.48\textwidth}
        \textbf{Relatore:}\\
        Chiar.mo~Prof.~Alessandro~Depari
    \end{minipage}
    \begin{minipage}{.5\textwidth}
        \begin{flushright}
            \textbf{Laureando:}\\
            \insertauthor\\
            Matr. 727199
        \end{flushright}
    \end{minipage}
    \begin{center}
        Anno Accademico \insertdate
    \end{center}
}
\setbeamertemplate{frametitle continuation}{\hfill\insertcontinuationcount}
\setbeamertemplate{footline}[text line]{%
    
    \parbox{\linewidth}{\vspace*{-10pt}\insertrawshotauthor\hspace{5mm}\insertrawshorttitle\hfill\vspace{10pt}\hspace{5mm}\includegraphics[height=10mm]{logo_unibs_40.png}\insertframenumber}}
\setbeamertemplate{navigation symbols}{}

\setbeamercolor{background canvas}{bg=white}
\setbeamercolor{normal text}{fg=textcolor}
\setbeamercolor{frametitle}{bg=unibsblue, fg=white}

\hypersetup{
    pdftitle={Sviluppo di un sistema di supporto alla programmazione per dispositivi integrati della famiglia AVR},
    pdfsubject={RelazioneFinale},
    pdfauthor={Stefano Fontana},
    hidelinks
}

\usepackage{pgfpages}

%\pgfpagesuselayout{2 on 1}[a4paper, border shrink=8mm]

%\setbeameroption{show notes on second screen=bottom}

\newlength{\parskipbackup}
\setlength{\parskipbackup}{\parskip}
\newlength{\parindentbackup}
\setlength{\parindentbackup}{\parindent}
\newcommand{\baselinestretchbackup}{\baselinestretch}

\usetemplatenote{\rmfamily%
  \setlength{\parindent}{1em} \setlength{\parskip}{1ex}%
  \renewcommand{\baselinestretch}{1}%
  \noindent \insertnote%

  \setlength{\parskip}{\parskipbackup}%
  \setlength{\parindent}{\parindentbackup}%
  \renewcommand{\baselinestretch}{\baselinestretchbackup}%
}

\pgfpageslogicalpageoptions{1}{border code=\pgfusepath{stroke}}


\begin{document}

    \pagenumbering{gobble}
    \maketitle

    \begin{frame}
        \frametitle{Stato dell'arte}

        La programmazione di dispositivi integrati viene effettuata tramite tre macro passaggi

        \begin{itemize}
            \item<1-> Codifica \\
            {\footnotesize Il codice viene scritto dal programmatore e compilato in formato binario}
            \item<2-> Caricamento \\
            {\footnotesize Il programma viene scritto sulla memoria del controllore tramite l'uso di un programmatore}
            \item<3-> Execuzione\\
            {\footnotesize Il controllore viene quindi avviato e il programma viene eseguito.}
        \end{itemize}
    
    \end{frame}
    \note{
        Il programmatore converte un protocollo complesso (USB/seriale) in comandi adatti alla programmazione dell'integrato su bus parallelo/spi
    }

    \begin{frame}
        \frametitle{Stato dell'arte}

        Una volta avviata l'esecuzione, si passa alla verifica e alla validazione

        \begin{figure}
            \begin{tikzpicture}
                \draw[fill=green!50] (0,0) rectangle (2,2) node[pos=.5] {\textit{host}};
                \draw[fill=yellow!50] (3,0.5) rectangle (4,1.5) node[pos=.5] {Programmatore};
                \draw[fill=blue!50] (5,0.5) rectangle (6,1.5) node[pos=.5] {\textit{target}};
            \end{tikzpicture}
        \end{figure}

    \end{frame}
    \note{
        
    }


    \begin{frame}[allowframebreaks]
        \frametitle{Problematiche comuni}
        ciao
        \framebreak
        
        ciao
    \end{frame}
    \note{
        \begin{itemize}
            \item Problema di utilizzo comunicazione seriale
            \item Problema di reupload
        \end{itemize}
    }


    \begin{frame}
        \frametitle{Architettura GDB}

    \end{frame}
    \note{
        Viene introdotta l'architettura di debugging
    }


    \begin{frame}
        \frametitle{Real Time Terminal}

    \end{frame}
    \note{
    }


    \begin{frame}
        \frametitle{Modifiche hardware}

    \end{frame}
    \note{
        Motivazioni per la scelta\\
        Vengono illustrate le modifiche all'hardware di arduino Uno
    }


    \begin{frame}
        \frametitle{Ambiente di sviluppo}

    \end{frame}
    \note{
        Illustrazione di vscode in modalità di debugging
    }

    
    
\end{document}
